\chapter{Conclusions and future work}

\section{Conclusion and Future Work}

This thesis has presented the comprehensive calibration and measurement of luminosity for the 2023 data-taking period of the CMS experiment at the LHC. The luminosity calibration was conducted using the van der Meer (vdM) scan methodology, which included meticulous corrections for several systematic effects. Among these, the beam-beam interaction, XY factorization bias, and scan-to-scan vdM variation were identified as the most significant sources of uncertainty. Despite these challenges, the final calibration achieved an impressive consistency across all independently calibrated luminometers within a 0.16\% uncertainty during the head-on period of the vdM fill. 

The extrapolation of the vdM calibration to physics conditions required additional stability and linearity corrections and detector specific corrections like afterglow and out-of-time pileup. These corrections significantly reduced the stability uncertainty from an initial 2.6\% to 0.71\%, with residual non-linearity contributing an additional 0.51\% uncertainty. As a result, the total delivered integrated luminosity for the 2023 data-taking year was measured to be 32.74 $\text{fb}^{-1}$, with an associated overall uncertainty of 1.28\%. This represents the best preliminary luminosity results ever achieved by CMS, marking a significant milestone in the precision of luminosity measurements.

A key part of my contributions focused on enhancing the performance of the vdM scan data analysis conducted by the VdMFramework. After identifying a critical input/output (I/O) bottleneck, which accounted for approximately 75\% of the total execution time, the implementation of caching techniques led to a fourfold increase in efficiency, significantly speeding up the analysis. Further improvements were made to the automatic online scan analysis workflow with the introduction of a file polling mechanism, which reduced the dead time by a factor of 20, enabling the entire scan analysis pipeline to complete in approximately one minute.

In addition to these performance enhancements, substantial contributions were made to the BRIL Work Suite, particularly in improving the usability and maintainability of the \textit{iovtag} structure. By making the payload parameters more transparent and integral to the calibration process, the updates allowed for a more intuitive and error-resistant workflow. Moreover, an alternative tool was developed to work alongside the BRIL Work Suite, providing a means to circumvent lengthy database upload times. This tool allowed for quicker iterations during the analysis process and enabled cross-verification of the data by comparing database content with raw HDF5 files.

While significant progress has been made, there remains much work to be done. The largest contributions to the overall luminosity uncertainty stem from stability and linearity corrections, marking them as the primary focus for future improvements. On the software side, optimizing the alternative to \textit{brilcalc} promises the most immediate gains, as it will streamline the analysis workflow further. Additionally, optimization efforts will continue for Stages II and III of the VdMFramework, where data corrections and fits are applied. Ultimately, these efforts aim to reduce the overall uncertainty of the 2023 luminosity measurement, bringing it closer to the long-term goal of reaching the challenging 1\% uncertainty target.

In conclusion, this work has not only advanced the precision of luminosity measurements at CMS but also set a new benchmark as the best preliminary luminosity results achieved by the experiment. By continuing to refine both the calibration techniques and the supporting software infrastructure, the goal of achieving a more precise and accurate luminosity measurement is ever closer.
