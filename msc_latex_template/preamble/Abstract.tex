\chapter*{Abstract}

The quest to understand the fundamental nature of the universe drives the field of particle physics, centered around the Standard Model (SM), which describes the fundamental particles and three of the four known forces: electromagnetism, the weak force, and the strong force. Despite its success in explaining a wide range of phenomena, the SM does not address some of the most profound questions in physics, such as the nature of dark matter, the origin of mass hierarchies, and the unification of forces. Probing the SM with precision measurements is essential for uncovering potential new physics beyond the current theoretical framework.

Particle colliders like the Large Hadron Collider (LHC) at CERN are at the forefront of this exploration, where measurements of cross sections are a crucial part of the physics program that tests the SM’s predictions. A cross section quantifies the likelihood of specific particle's interactions, and accurate luminosity measurements are fundamental to determining these cross sections. Luminosity, \(\mathcal{L}\), is proportional to the rate of collisions, and thus any uncertainty in \(\mathcal{L}\) translates into uncertainties in the cross sections, affecting the reliability of the conclusions drawn from experimental data.

This thesis presents the preliminary calibration and measurement of luminosity for the 2023 proton-proton collision data-taking period at a center-of-mass energy of 13.6 TeV with the CMS experiment. The absolute luminosity scale is established using the van der Meer (vdM) scan methodology, a technique that involves scanning the particle beams across each other to measure their overlap and calibrate the visible cross sections of the luminometers. A series of detectors were employed to provide real-time luminosity measurements, with systematic corrections and stability analyses ensuring consistent and accurate results

The final integrated luminosity was measured as 32.74 fb\textsuperscript{-1} with a total uncertainty of 1.28\%, representing the best preliminary luminosity results achieved by CMS. In addition to the experimental efforts, substantial software improvements were made to the codebases used in the analysis. These advancements enhanced data processing efficiency and streamlined the calibration workflow, enabling more rapid and reliable analysis iterations.

\paragraph{Keywords} CERN, CMS, Luminosity, van der Meer

\cleardoublepage

\chapter*{Resumo}

A busca por compreender a natureza fundamental do universo impulsiona o campo da física de partículas, centrado no Modelo Padrão (SM), que descreve as partículas fundamentais e três das quatro forças conhecidas: o electromagnetismo, a força fraca e a força forte. Apesar do seu sucesso em explicar uma ampla gama de fenómenos, o SM não aborda algumas das questões mais profundas da física, como a natureza da matéria escura, a origem das hierarquias de massa e a unificação das forças. Testar o SM com medições de alta precisão é essencial para descobrir novas possíveis físicas para além do atual quadro teórico.

Colisores de partículas como o Grande Colisor de Partículas (LHC) no CERN estão na vanguarda desta exploração, onde as medições das secções eficazes são cruciais para testar as previsões do SM. A secção eficaz quantifica a probabilidade de interações entre específicas partículas, e medições precisas da luminosidade são fundamentais para determinar estas secções eficazes. A luminosidade, \(\mathcal{L}\), influencia diretamente a taxa de colisões, e, portanto, qualquer incerteza em \(\mathcal{L}\) traduz-se numa incerteza nas secções eficazes, afetando a fiabilidade das conclusões extraídas dos dados experimentais.

Esta tese foca-se na calibração preliminar e medição da luminosidade para o período de 2023 que corresponde a colisões protão-protão a uma energia de centro de massa de 13,6 TeV na experiência CMS. A escala absoluta da luminosidade é estabelecida utilizando a metodologia de scans de van der Meer (vdM), uma técnica que baseada no scan dos feixes de partículas entre si para medir a sobreposição e calibrar as secções eficazes visíveis dos luminómetros. Foi empregue uma série de detectores para fornecer medições de luminosidade em tempo real, com correções sistemáticas e análises de estabilidade a garantir resultados consistentes e precisos.

A medição final da luminosidade integrada foi de 32.74 fb\textsuperscript{-1} com uma incerteza total de 1.28\%, o melhor resultado preliminar alcançado pelo CMS. Além dos esforços experimentais, foram feitas melhorias no software utilizado nas análises. Estes avanços aumentaram a eficiência do processamento de dados e simplificaram o fluxo de trabalho da calibração, permitindo iterações de análise mais rápidas e fiáveis.

\paragraph{Palavras-chave} CERN, CMS, Luminosidade, van der Meer

\cleardoublepage
